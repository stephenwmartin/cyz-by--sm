\chapter{Algunos recuerdos}
\texttt{Viernes 10 de Febrero 2038 @ 10:00 AM}
\vspace*{1.5cm}
Realmente nunca me he considerado un buen escritor, es decir, tuve algunas notas sobresalientes en la escuela, 
pero vamos, es la escuela. Escribes un ensayo aquí y allá y si tienes suerte, tu profesor tal vez le echará un ojo.

Y no es que pretenda serlo, simplemente quiero dejar evidencia de todo lo que me va pasando, de lo que voy 
sintiendo y todas esas cosas que nos recuerdan que somos humanos en situaciones como estas, y que los humanos somos demasiado frágiles, aunque claro, 
no creo que tu, el que lee haya pasado por una situación remotamente parecida a esta. O sí.\\
Escribo esto por si algún día esto ha terminado y alguien puede leerlo. De verdad lo espero.\\
Extraño el ir al colegio, los buses llenos de gente apurada, el estrés, incluso extraño eso, pero lo que mas extraño es estar con mi familia, pelear con mis hermanos, charlar con mi padre, Dios, abrazar a mi madre.\\
Pero eso no es posible, al menos no fuera de mi mente, aunque en algún lugar de mi cerebro, a veces se 
enciende una pequeña luz de esperanza, esperanza de volver a verlos, de que hubieran salido antes y  se salvaran, esperanza de que estén refugiados en alguno de los albergues, en un punto seguro, no se, es bastante improbable pero el humano siempre tendrá esas pequeñas motas de esperanza en los momentos mas desoladores.\\
Algo que me duele de sobremanera es no tener un último recuerdo dulce de mi madre, pelee con ella y, ahora que lo pienso fue una estupidez, característica de la adolescencia. Esperaba que al regresar a casa todo se arreglara, platicaramos y terminaramos riendo. Pero no fue así, desde ese día no la he vuelto a ver.\\

Con mi padre fue diferente, aunque la última vez que hablé con el fue por teléfono, pero fue una llamada muy 
emotiva. Supongo que eso no importa ahora, lo único que importa ahora es sobrevivir...\\
Sobrevivir y encontrarla.
