\documentclass[9pt]{memoir}
\usepackage{fontspec}

\usepackage[osf]{mathpazo}
\usepackage[spanish]{babel}
\usepackage[mathcal,mathbf]{euler}
\usepackage{amsmath,amssymb,amsthm}
\usepackage{graphicx,sidecap,tikz}
\usepackage{siunitx} % automatic number formatting, decimal point alignment
\usepackage[hidelinks]{hyperref}

% To get lining figures in tables set by siunitx, which apparently uses the
% \mathrm font instead of \mathnormal
\SetMathAlphabet{\mathrm}{normal}{U}{eur}{m}{n}

% =========================
% = Setting up the layout =
% =========================

% With a 9pt body font we want a little extra line spacing (I mean *leading*)
\setSingleSpace{1.2}
\SingleSpacing

% Okay, holy crap. Calculating the correct type block height by hand is quite
% challenging (partially because not all lines are \baselineskip high;
% apparently the first line is \topskip high?), and \checkandfixthelayout will
% in the end actually *change* it so that the type block is always an integer
% multiple of lines. The easiest thing is to set the approximate desired type
% block height, the width, the left or right margin, the bottom margin, and
% the headdrop, and then let memoir take care of everything else. Changing the
% algorithm used to check the layout helps as well.
\setstocksize{9in}{6in}
\settrimmedsize{\stockheight}{\stockwidth}{*}
\settrims{0pt}{0pt}

\settypeblocksize{46\baselineskip}{4in}{*}
\setlrmargins{*}{0.5in}{*}
\setulmargins{*}{0.5in}{*}

\setheadfoot{\baselineskip}{\baselineskip} % headheight and footskip
\setheaderspaces{0.5in}{*}{*} % headdrop, headsep, and ratio

\checkandfixthelayout[lines]

% Set up custom headers and footers
\makepagestyle{stylish}
\copypagestyle{stylish}{headings}
\makerunningwidth{stylish}{5in}
\makeheadposition{stylish}{flushleft}{flushright}{}{}
\pagestyle{stylish}

% ============================
% = Table of contents tweaks =
% ============================
\renewcommand*{\contentsname}{Table of Contents}
\setsecnumdepth{subsubsection}
\settocdepth{subsection}

% ============
% = Chapters =
% ============
\newcommand{\swelledrule}{%
    \tikz \filldraw[scale=.015,very thin]%
    (0,0) -- (100,1) -- (200,1) -- (300,0) --%
    (200,-1) -- (100,-1) -- cycle;}
\makeatletter
\makechapterstyle{grady}{%
    \setlength{\beforechapskip}{0pt}
    \renewcommand*{\chapnamefont}{\large\centering\scshape}
    \renewcommand*{\chapnumfont}{\large}
    \renewcommand*{\printchapternum}{%
        \chapnumfont \ifanappendix \thechapter \else \numtoName{\c@chapter}\fi}
    \renewcommand*{\printchapternonum}{%
        \vphantom{\printchaptername}%
        \vphantom{\chapnumfont 1}%
        \afterchapternum
        \vskip -\onelineskip}
    \renewcommand*{\chaptitlefont}{\Huge\itshape}
    \renewcommand*{\printchaptertitle}[1]{%
        \centering\chaptitlefont ##1\par\swelledrule}}
\makeatother
\chapterstyle{grady}

% See below, after introduction, for \clearforchapter

% Prevent page numbers from appearing on chapter pages
\aliaspagestyle{chapter}{empty}

% ===================
% = Marginal labels =
% ===================
\strictpagecheck % Turn on robust page checking
\captiondelim{} % Don't print a colon after "Figure #.#"

\makeatletter
\renewcommand{\fnum@figure}{%
    \checkoddpage%
    \ifoddpage%
        \makebox[0pt][l]{\hspace{-1in}{\scshape\figurename~\thefigure}}%
    \else
        \makebox[0pt][r]{{\scshape\figurename~\thefigure}\hspace*{-5in}}%
    \fi%
    }
\renewcommand{\fnum@table}{%
    \checkoddpage%
    \ifoddpage%
        \makebox[0pt][l]{\hspace{-1in}{\scshape\tablename~\thetable}}%
    \else
        \makebox[0pt][r]{{\scshape\tablename~\thetable}\hspace*{-5in}}%
    \fi%
    }
\let\mytagform@=\tagform@
\def\tagform@#1{%
\checkoddpage%
    \ifoddpage%
    \makebox[1sp][l]{\hspace{-5in}\maketag@@@{(\ignorespaces#1 \unskip \@@italiccorr)}}%
    \else
    \makebox[1sp][r]{\maketag@@@{(\ignorespaces#1 \unskip \@@italiccorr)}\hspace*{-1in}}%
    \fi%
    }
\renewcommand{\eqref}[1]{\textup{\mytagform@{\ref{#1}}}}
\makeatother

\usetikzlibrary{arrows,positioning,decorations.pathmorphing,trees}

\begin{document}

\frontmatter
\thispagestyle{empty}

\mbox{}\vspace{2in}
\noindent
\begin{flushright}
{\LARGE Cianuro y zombies\\[2\baselineskip]}
\end{flushright}

\vspace{6\baselineskip}
\hfill{\Large\scshape{}Stephen W. Martin}

\cleartorecto\tableofcontents*

\mainmatter

\chapter{Algunos recuerdos}
\texttt{Viernes 10 de Febrero 2038 @ 10:00 AM}
\vspace*{1.5cm}
Realmente nunca me he considerado un buen escritor, es decir, tuve algunas notas sobresalientes en la escuela, 
pero vamos, es la escuela. Escribes un ensayo aquí y allá y si tienes suerte, tu profesor tal vez le echará un ojo.

Y no es que pretenda serlo, simplemente quiero dejar evidencia de todo lo que me va pasando, de lo que voy 
sintiendo y todas esas cosas que nos recuerdan que somos humanos en situaciones como estas, y que los humanos somos demasiado frágiles, aunque claro, 
no creo que tu, el que lee haya pasado por una situación remotamente parecida a esta. O sí.\\
Escribo esto por si algún día esto ha terminado y alguien puede leerlo. De verdad lo espero.\\
Extraño el ir al colegio, los buses llenos de gente apurada, el estrés, incluso extraño eso, pero lo que mas extraño es estar con mi familia, pelear con mis hermanos, charlar con mi padre, Dios, abrazar a mi madre.\\
Pero eso no es posible, al menos no fuera de mi mente, aunque en algún lugar de mi cerebro, a veces se 
enciende una pequeña luz de esperanza, esperanza de volver a verlos, de que hubieran salido antes y  se salvaran, esperanza de que estén refugiados en alguno de los albergues, en un punto seguro, no se, es bastante improbable pero el humano siempre tendrá esas pequeñas motas de esperanza en los momentos mas desoladores.\\
Algo que me duele de sobremanera es no tener un último recuerdo dulce de mi madre, pelee con ella y, ahora que lo pienso fue una estupidez, característica de la adolescencia. Esperaba que al regresar a casa todo se arreglara, platicaramos y terminaramos riendo. Pero no fue así, desde ese día no la he vuelto a ver.\\

Con mi padre fue diferente, aunque la última vez que hablé con el fue por teléfono, pero fue una llamada muy 
emotiva. Supongo que eso no importa ahora, lo único que importa ahora es sobrevivir...\\
Sobrevivir y encontrarla.


\chapter{Algo no anda bien}
\textbf{\textit{Sábado 11 de Febrero 2038 @ 9:30 AM}}\\\\[1.5cm]
No puedo quejarme, realmente la he podido librar bastante bien, estoy en un refugio que parece bastante 
seguro, aunque pensándolo bien, eso decían de los que implementó el ejército, es inquietante pensar en eso.
Tengo comida para al menos 1 mes, estos tipos estaban bien abastecidos, se los agradezco, dondequiera que 
estén, y, Dios los ayude si ya son unas de esas cosas.\\
En fin, no quiero pensar mucho en que tendré que hacer cuando se terminen las provisiones, pero supongo que 
en algún punto tendré que hacerlo.
Mientras tanto trato de recordar, aunque duela.\\\\
Aquí voy.
\newpage

\textbf{\textit{Miércoles 12 de Diciembre 2037 @ 3:00 PM\\\\[1.5cm]}}
Mis padres me compraron una nueva laptop, y ahora me dispongo a hacer un diario sobre todo lo que vive un 
adolescente de 17 años, o casi todo.\\
En la mañana tuve una interesante discusión con mi mejor amiga, Victoria, aunque ella prefiere que le llame
Vick, de hecho siempre que le llamo Victoria pone los ojos en blanco y me mira con actitud de bronca. En fin, platicábamos sobre lo que nos dijo un profesor del colegio, un tipo bastante gordo pero muy intelectual, 
Vai su apellido ,el dice que en algún punto, las potencias mundiales se atacarán con cosas biológicas, cosas que harán a la gente volverse locos y comerse unos a otros. El tipo es un poco loco, de eso platiqué con mi amiga, aunque al final concluimos que podría ser posible.\\\\

\textit{\textbf{Viernes 14 de Diciembre 2037 @ 1:00 PM\\\\[1.5cm]}}
Vick me habló en la tarde de ayer, encontró un blog (\url{zombiesatemygirl.blogspot.com}) con historias de zombies 
obviamente, leí unas cuantas, me gusta el tema, pero pensar que pudiera volverse realidad, como lo que dice el maestro Vai, me da un poco de escalofríos. Aún así Vick y yo, intercambiamos historias que encontrábamos ahí hasta la madrugada, así que ahora tengo que lidiar con un sueño del demonio por que tengo un examen de 
inglés mañana y no he estudiado mucho.\\
Lore dijo que me ayudaría a estudiar pero creo que no es buena idea, creo que esa chica está un poco loca pero bueno, es eso o reprobar ese examen, lo cual después de la plática con mi madre donde me dió un ultimátum para aprobar ese nivel no se buena idea.\\\\

\textbf{\textit{Domingo 15 de Diciembre 2037 6:00 PM\\[1.5cm]}}
Me puse a leer mas historias de zombies, son fascinantes, de hecho, creo que tengo un poco hartos a mis amigos de Facebook, esos botones de «compartir historia» de la página son un gran invento, al menos para Vick y para mi.\\
Es agradable tener una amiga así, que tenga los mismo intereses y con la cual puedes hablar de lo que sea, 
Vick es una de ellas...\\
Pero bueno, regresando al asunto del que nos hablaba Vai, esta mañana tenía una nota en mi feed bastante inquietante.\\
«Wikileaks dice que conoce un secreto bastante sucio sobre bases militares en Arabia Saudita...»- Empezaba el artículo, y es que según el autor, el cable de Wikileaks había dicho que tenía en su poder documentos 
pertenecientes a una base miltar ultra secreta en Arabia Saudita, en los que según ellos se describía un plan 
de contingencia ante una amenaza biológica en Estados Unidos, al parecer es una base secreta de la CIA, y 
aunque el artículo no dice nada mas sobre ello, deja al lector bastante inquieto sobre el tema. Le mandé por 
correo el artículo a Vai y lo único que me respondió fue: «Te lo dije.».\\
Le conté a Vick y a Joel, Joel dice que no hay por que preocuparse por que el artículo también dice que esos 
planes eran de hace 50 años. La teoría de Vick es que todo sigue vigente.\\
Inquietante, ¿no? \\\\


\textbf{\textit{Domingo 12 de Febrero 2038 @ 2:30 PM}}\\\\[1.5cm]

Como dije estoy refugiado en lo que parece ser un edificio de gobierno, he explorado casi todo y al parecer está vacío, lo único fuera de mi alcance es un sótano cerrado con candado, pero, he estado fuera de él con el oído pegado a la puerta y parece estar vacío, no se escucha ni un sólo ruido, no es que importe mucho.\\ He empujado un mueble que hacía de archivero hace tiempo, pesaba una tonelada como mínimo, creo que servirá para atorar las puertas de la entrada principal.
Hay una radio de onda corta en uno de los cuartos, las baterías estaban muertas, pero buscando en un estante encontré un paquete nuevo y ahora solo escucho silencio. Maldito silencio.\\
En fin, algo que me inquieta de este lugar es que está bastante ordenado, es como si no se hubiera enterado que hubo un apocalipsis afuera.\\
Me he pasado la mayor parte de la mañana viendo una fotografía de nosotros tres y, claro llorando como un bebé. La tomaron en el cumpleaños 16 de Vick, nos vemos tan felices, con ilusión de que vendría algo mejor. Dios, esto es horrible, no puedo seguir, escribiré luego, las lágrimas vuelven a bajar por mis mejillas.\\[4cm]
Ya logré calmarme, estar solo no lleva a nada bueno, la tristeza es sobrecogedora y los recuerdos no ayudan en nada. Además de llorar como si siguiera en kinder he explorado mas el edificio, encontré una Beretta 92 y municiones en uno de los cuartos del segundo piso, estaba oculta en un cajón con llave, la cual estaba sobre el escritorio, ignoro el por qué. Reconocería esa pistola en cualquier lugar, mi padre era General, así que es natural que yo conozca de armas.




\end{document}