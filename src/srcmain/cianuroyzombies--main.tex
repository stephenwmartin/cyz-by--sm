\documentclass[9pt]{memoir}
\usepackage{fontspec}

\usepackage[osf]{mathpazo}
\usepackage[spanish]{babel}
\usepackage[mathcal,mathbf]{euler}
\usepackage{amsmath,amssymb,amsthm}
\usepackage{graphicx,sidecap,tikz}
\usepackage{siunitx} % automatic number formatting, decimal point alignment
\usepackage[hidelinks]{hyperref}

% To get lining figures in tables set by siunitx, which apparently uses the
% \mathrm font instead of \mathnormal
\SetMathAlphabet{\mathrm}{normal}{U}{eur}{m}{n}

% =========================
% = Setting up the layout =
% =========================

% With a 9pt body font we want a little extra line spacing (I mean *leading*)
\setSingleSpace{1.2}
\SingleSpacing

% Okay, holy crap. Calculating the correct type block height by hand is quite
% challenging (partially because not all lines are \baselineskip high;
% apparently the first line is \topskip high?), and \checkandfixthelayout will
% in the end actually *change* it so that the type block is always an integer
% multiple of lines. The easiest thing is to set the approximate desired type
% block height, the width, the left or right margin, the bottom margin, and
% the headdrop, and then let memoir take care of everything else. Changing the
% algorithm used to check the layout helps as well.
\setstocksize{9in}{6in}
\settrimmedsize{\stockheight}{\stockwidth}{*}
\settrims{0pt}{0pt}

\settypeblocksize{46\baselineskip}{4in}{*}
\setlrmargins{*}{0.5in}{*}
\setulmargins{*}{0.5in}{*}

\setheadfoot{\baselineskip}{\baselineskip} % headheight and footskip
\setheaderspaces{0.5in}{*}{*} % headdrop, headsep, and ratio

\checkandfixthelayout[lines]

% Set up custom headers and footers
\makepagestyle{stylish}
\copypagestyle{stylish}{headings}
\makerunningwidth{stylish}{5in}
\makeheadposition{stylish}{flushleft}{flushright}{}{}
\pagestyle{stylish}

% ============================
% = Table of contents tweaks =
% ============================
\renewcommand*{\contentsname}{Table of Contents}
\setsecnumdepth{subsubsection}
\settocdepth{subsection}

% ============
% = Chapters =
% ============
\newcommand{\swelledrule}{%
    \tikz \filldraw[scale=.015,very thin]%
    (0,0) -- (100,1) -- (200,1) -- (300,0) --%
    (200,-1) -- (100,-1) -- cycle;}
\makeatletter
\makechapterstyle{grady}{%
    \setlength{\beforechapskip}{0pt}
    \renewcommand*{\chapnamefont}{\large\centering\scshape}
    \renewcommand*{\chapnumfont}{\large}
    \renewcommand*{\printchapternum}{%
        \chapnumfont \ifanappendix \thechapter \else \numtoName{\c@chapter}\fi}
    \renewcommand*{\printchapternonum}{%
        \vphantom{\printchaptername}%
        \vphantom{\chapnumfont 1}%
        \afterchapternum
        \vskip -\onelineskip}
    \renewcommand*{\chaptitlefont}{\Huge\itshape}
    \renewcommand*{\printchaptertitle}[1]{%
        \centering\chaptitlefont ##1\par\swelledrule}}
\makeatother
\chapterstyle{grady}

% See below, after introduction, for \clearforchapter

% Prevent page numbers from appearing on chapter pages
\aliaspagestyle{chapter}{empty}

% ===================
% = Marginal labels =
% ===================
\strictpagecheck % Turn on robust page checking
\captiondelim{} % Don't print a colon after "Figure #.#"

\makeatletter
\renewcommand{\fnum@figure}{%
    \checkoddpage%
    \ifoddpage%
        \makebox[0pt][l]{\hspace{-1in}{\scshape\figurename~\thefigure}}%
    \else
        \makebox[0pt][r]{{\scshape\figurename~\thefigure}\hspace*{-5in}}%
    \fi%
    }
\renewcommand{\fnum@table}{%
    \checkoddpage%
    \ifoddpage%
        \makebox[0pt][l]{\hspace{-1in}{\scshape\tablename~\thetable}}%
    \else
        \makebox[0pt][r]{{\scshape\tablename~\thetable}\hspace*{-5in}}%
    \fi%
    }
\let\mytagform@=\tagform@
\def\tagform@#1{%
\checkoddpage%
    \ifoddpage%
    \makebox[1sp][l]{\hspace{-5in}\maketag@@@{(\ignorespaces#1 \unskip \@@italiccorr)}}%
    \else
    \makebox[1sp][r]{\maketag@@@{(\ignorespaces#1 \unskip \@@italiccorr)}\hspace*{-1in}}%
    \fi%
    }
\renewcommand{\eqref}[1]{\textup{\mytagform@{\ref{#1}}}}
\makeatother

\usetikzlibrary{arrows,positioning,decorations.pathmorphing,trees}

\begin{document}

\frontmatter
\thispagestyle{empty}

\mbox{}\vspace{2in}
\noindent
\begin{flushright}
{\LARGE Cianuro y zombies\\[2\baselineskip]}
\end{flushright}

\vspace{6\baselineskip}
\hfill{\Large\scshape{}Stephen W. Martin}

\cleartorecto\tableofcontents*

\mainmatter

\chapter{Algunos recuerdos}
\vspace*{1.5cm}
\section*{\texttt{Viernes 10 de Febrero 2038 @ 10:00 AM}}
\vspace*{1.5cm}
Realmente nunca me he considerado un buen escritor, es decir, tuve algunas notas sobresalientes en la escuela, 
pero vamos, es la escuela. Escribes un ensayo aquí y allá y si tienes suerte, tu profesor tal vez le echará un ojo.

Y no es que pretenda serlo, simplemente quiero dejar evidencia de todo lo que me va pasando, de lo que voy 
sintiendo y todas esas cosas que nos recuerdan que somos humanos en situaciones como estas, y que los humanos somos demasiado frágiles, aunque claro, 
no creo que tu, el que lee haya pasado por una situación remotamente parecida a esta. O sí.

Escribo esto por si algún día esto ha terminado y alguien puede leerlo. De verdad lo espero.

Extraño el ir al colegio, los buses llenos de gente apurada, el estrés, incluso extraño eso, pero lo que mas extraño es estar con mi familia, pelear con mis hermanos, charlar con mi padre, Dios, abrazar a mi madre.

Pero eso no es posible, al menos no fuera de mi mente, aunque en algún lugar de mi cerebro, a veces se 
enciende una pequeña luz de esperanza, esperanza de volver a verlos, de que hubieran salido antes y  se salvaran, esperanza de que estén refugiados en alguno de los albergues, en un punto seguro, no se, es bastante improbable pero el humano siempre tendrá esas pequeñas motas de esperanza en los momentos mas desoladores.

Algo que me duele de sobremanera es no tener un último recuerdo dulce de mi madre, pelee con ella y, ahora que lo pienso fue una estupidez, característica de la adolescencia. Esperaba que al regresar a casa todo se arreglara, platicaramos y terminaramos riendo. Pero no fue así, desde ese día no la he vuelto a ver.

Con mi padre fue diferente, aunque la última vez que hablé con el fue por teléfono, pero fue una llamada muy 
emotiva. Supongo que eso no importa ahora, lo único que importa ahora es sobrevivir...

Sobrevivir y encontrarla.


\input{../cap2AlgoNoAndaBien/capitulo2--algoNoAndaBien}


\end{document}